\chapter{Implementation}
\lstset{language=Python, captionpos=b, frame=single}
\captionsetup{width=.8\linewidth} 


Guess what? At the beginning of each chapter, a description should introduce the reader to the content of the chapter. The description should explain to the reader the layout of the chapter, the contribution that the chapter makes to the overall dissertation and the contribution of the individual sections towards the overall chapter.


\section{Overview of the Solution}

%% Short caption for the table of listings - long caption for the explanation for the reader
\includecode{Sample Code}{Lengthy caption explaining the code to the reader}{lst:snippet}{snippet.py}

The code in listing~\ref{lst:snippet} is a demonstration how to include a file with code into the template.



\section{Component One}

%% Defaults for listings

The code in listing~\ref{lst:snippet2} is a demonstration how to include code in the template.

%% Short caption for the table of listings - long caption for the explanation for the reader
\begin{lstlisting}[caption={[Sample Code 2]Second Lengthy caption}, label={lst:snippet2}]
x = 1
if x == 1:
    # indented four spaces
    print("x is 1.")
\end{lstlisting}


\section{Summary}

Every chapter aside from the first and last chapter should conclude with a summary. 

