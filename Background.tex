\chapter{State of the Art}

At the beginning of each chapter, a description should introduce the reader to the content of the chapter. The description should explain to the reader the layout of the chapter, the contribution that the chapter makes to the overall dissertation and the contribution of the individual sections towards the overall chapter.

From the perspective of this document forming part of your degree, this chapter should demonstrate to the reader your knowledge of the area of your dissertation project. It should present your knowledge in a coherent and detailed form. The reader should understand that you have in-depth knowledge of the area of the dissertation without being overloaded with information.

\section{Background}

A section on the background of the dissertation should provide the reader with an introduction to existing technologies and concepts that form the basis of the
work presented in the dissertation.


\section{Closely-Related Work}

Work in research areas tends to address a number of specific aspects. Ideally, the discussion of published research should focus on the aspects that have been addressed by various publications - and not a discussion of the individual publications.

For example, if the topic would be a discussion of work on programming languages, the subsections of the related work could be discussions of object orientation and its realisation in various languages or the use of lambda functions by these languages.

\subsection{Aspect \#1}


\subsection{Aspect \#2}

\section{Summary}

Summarize the chapter and present a comparison of the projects that you reviewed.

\begin{table}[!h]
\begin{center}
	\begin{tabular}{|l|c|c|} 
	\hline
 	\bf  & \bf Aspect \#1  & \bf Aspect \#2 \\
  	\hline
	Row 1 & Item 1 & Item 2 \\
	Row 2 & Item 1 & Item 2 \\
	Row 3 & Item 1 & Item 2 \\
	Row 4 & Item 1 & Item 2 \\
	\hline
	\end{tabular}
\end{center}
\caption[Comparison of Closely-Related Projects]{Caption that explains the table to the reader}	
\label{tab:SummaryProjects}
\end{table}
