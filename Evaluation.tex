\chapter{Evaluation}
\label{chap:Evaluation}

The Evaluation chapter should present a comparison of the work that forms the basis of the dissertation and existing work. At a higher level, it should demonstrate an awareness of the relationship of the dissertation work to the research area that it is based in.

\section{Experiments}

In the case where experiments have been carried out, the experimental setup and the values that were defined for the variables need to be presented in a table e.g. table~\ref{tab:experimentsetup}.

\begin{table}[!h]
\begin{center}
	\begin{tabular}{|l|c|c|} 
	\hline
 	\bf Column 1  & \bf Column 2  & \bf Column 3 \\
  	\hline
	Row 1 & Item 1 & Item 2 \\
	Row 2 & Item 1 & Item 2 \\
	Row 3 & Item 1 & Item 2 \\
	Row 4 & Item 1 & Item 2 \\
	\hline
	\end{tabular}
\end{center}
\caption[Variables of the experiment]{Caption that explains the table to the reader}	
\label{tab:experimentsetup}
\end{table}


\section{Results}

Figures that present results such as figure~\ref{fig:measurements} need to display descriptions of the axes, the units and scales of the measurements, statistical values, etc. Where measurements were taken from experiments, error bars or confidence intervals need to be provided to give the reader an indication of the spread of the measurements.

\includescalefigure{fig:measurements}{Measurement of System Wakeups}{Long caption that describes the figure to the reader}{1}{measurements.png}


\section{Summary}

Every chapter aside from the first and last chapter should conclude with a summary that presents the outcome of the chapter in a short, accessible form. 