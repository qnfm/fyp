\chapter{Design}
\label{chap:Design}

As outlined in Sections I and II, the WireGuard handshake is heavily based on DH, which does not have an efficient and well established post-quantum equivalent. Hence, in this section we describe how we replace DH by KEMs, for which well-established, efficient post-quantum instantiations exist.
We start by considering a simplified view on the core of the DH-based WireGuard handshake.


\section{Secuirt Aspect}
\label{sec:ProblemFormulation}
The primary goal of the adaptations proposed in this thesis is to provide security against post quantum attackers, who already are probably recording traffic at the current time for later decryption. Therefore, the necessary security properties that need to be guaranteed are listed below

\subsection{Secrecy}
Three key encapsulations using the keypairs \texttt{sski}/\texttt{spki},
\texttt{sskr}/\texttt{spkr}, and \texttt{eski}/\texttt{epki} provide
secrecy. Their respective ciphertexts are called \texttt{scti},
\texttt{sctr}, and \texttt{ectr} and the resulting keys are called
\texttt{spti}, \texttt{sptr}, \texttt{epti}. A single secure
encapsulation is sufficient to provide secrecy. We use two different
KEMs: Kyber and Classic McEliece.

\subsection{Authenticity}
The key encapsulation using the keypair \texttt{sskr}/\texttt{spkr}
authenticates the responder from the perspective of the initiator. The KEM encapsulation \texttt{sski}/\texttt{spki} authenticates the
initiator from the perspective of the responder. Authenticity is based
on the security of Classic McEliece alone.
\subsection{Pre-Shared Symmetric Key}
We allow the use of a pre-shared key (\texttt{psk}) as protocol input.
Even if all asymmetric security primitives turn out to be insecure,
providing a secure \texttt{psk} will have KEM-WireGuard authenticate both
peers, and output a secure shared key.
\subsection{Forward Secrecy}
Forward secrecy refers to secrecy of past sessions in case all static
keys are leaked. Imagine an attacker recording the network messages sent
between two devices, developing an interest in some particular exchange,
~and stealing both computers in an attempt to decrypt that conversation.
By stealing the hardware, the attacker gains access to \texttt{sski},
\texttt{sskr}, and the symmetric secret \texttt{psk}. Since the
ephemeral keypair \texttt{eski}/\texttt{epki} is generated on the fly
and deleted after the execution of the protocol, it cannot be recovered
by stealing the devices, and thus, KEM-WireGuard provides forward secrecy.
Forward secrecy relies on the security of Kyber and on proper
zeroization, i.e., the implementation must erase all temporary
variables.
\subsection{Security against State Disruption Attacks}

Both WireGuard and PQ-WireGuard are vulnerable to state disruption attacks; they rely
on a timestamp to protect against replay of the first protocol message.
An attacker who can tamper with the local time of the protocol initiator
can inhibit future handshakes, rendering the initiator's static keypair practically useless. Due to the use of the insecure NTP
protocol, real-world deployments are vulnerable to this attack. Lacking a reliable way to detect retransmission,
we remove the replay protection mechanism and store the responder state
in an encrypted cookie called ``the biscuit'' instead. Since the
responder does not store any session-dependent state until the initiator
is interactively authenticated, there is no state to disrupt in an
attack.

Note that while KEM-WireGuard is secure against state disruption, using it
does not protect WireGuard against the attack. Therefore, the hybrid
KEM-WireGuard/WireGuard setup recommended for deployment is still
vulnerable.

\subsection{Cryptographic Building Blocks}
All symmetric keys and hash values used in KEM-WireGuard are 32 bytes long.
\subsubsection{Hash}

A keyed hash function with one 32-byte input, one variable-size input,
and one 32-byte output. As keyed hash function we use the HMAC
construction {[}@rfc\_hmac{]} with BLAKE2s {[}@rfc\_blake2{]} as the
inner hash function.
\definecolor{gray}{rgb}{0.5,0.5,0.5}
\begin{tcolorbox}[colback=gray!]
\NormalTok{hash(key, data) {-}\textgreater{} key}
\end{tcolorbox}

\subsubsection{XAEAD}

Authenticated encryption with additional data for use with random
nonces. We use XChaCha20Poly1305 {[}@draft\_xchachapoly{]} in the
implementation, a construction also used by WireGuard.


\begin{tcolorbox}[colback=gray!]
\NormalTok{hash(key, data) {-}\textgreater{} key}
\NormalTok{XAEAD::enc(key, nonce, plaintext, additional\_data) {-}\textgreater{} ciphertext}
\NormalTok{XAEAD::dec(key, nonce, ciphertext, additional\_data) {-}\textgreater{} plaintext}
\end{tcolorbox}
\subsubsection{SKEM}
'Key Encapsulation Mechanism'(KEM) is the name of an interface widely
used in post-quantum-secure protocols. KEMs can be seen as asymmetric
encryption specifically for symmetric keys. KEM-WireGuard uses two different
KEMs. SKEM is the key encapsulation mechanism used with the static
keypairs in KEM-WireGuard. The public keys of these keypairs are not
transmitted over the wire during the protocol. We use Classic McEliece
460896 {[}@mceliece{]} which claims to be as hard to break as 192-bit
AES. As one of the oldest post-quantum-secure KEMs, it enjoys wide trust
among cryptographers, but it has not been chosen for standardization by
NIST. Its ciphertexts and private keys are small (188 bytes and 13568
bytes), and its public keys are large (524160 bytes). This fits our use
case: public keys are exchanged out-of-band, and only the small
ciphertexts have to be transmitted during the handshake.


\begin{tcolorbox}[colback=gray!]
\NormalTok{SKEM::enc(public\_key) {-}\textgreater{} (ciphertext, shared\_key)}
\NormalTok{SKEM::dec(secret\_key, ciphertext) {-}\textgreater{} shared\_key}
\end{tcolorbox}

\subsubsection{EKEM}

Key encapsulation mechanism used with the ephemeral KEM keypairs in
KEM-WireGuard. The public keys of these keypairs need to be transmitted over
the wire during the protocol. We use Kyber-512 {[}@kyber{]}, which has
been selected in the NIST post-quantum cryptography competition and
claims to be as hard to break as 128-bit AES. Its ciphertexts, public
keys, and private keys are 768, 800, and 1632 bytes long, respectively,
providing a good balance for our use case as both a public key and a
ciphertext have to be transmitted during the handshake.


\begin{tcolorbox}[colback=gray!]
\NormalTok{EKEM::enc(public\_key) {-}\textgreater{} (ciphertext, shared\_key)}
\NormalTok{EKEM::dec(secret\_key, ciphertext) {-}\textgreater{} shared\_key}
\end{tcolorbox}

Using a combination of two KEMs -- Classic McEliece for static keys and
Kyber for ephemeral keys -- results in large static public keys, but
allows us to fit all network messages into a single IPv6 frame.

KEM-WireGuard uses libsodium {[}@libsodium{]} as cryptographic backend for
hash, AEAD, and XAEAD, and liboqs {[}@liboqs{]} for the
post-quantum-secure KEMs.


\section{Overview of the Design}
\label{sec:OverviewOfDesign}
A description of the approach that addresses the problem identified above.
\begin{figure}[!htb]
\centering
\captionsetup{justification=centering}
\includestandalone{kem_protocol}
\caption{Generic KEM-WireGuard deisgn}
\end{figure}

\section{Summary}
\label{sec:SummaryDesign}

Every chapter aside from the first and last chapter should conclude with a summary. 