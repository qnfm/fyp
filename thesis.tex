%%%%%%%%%%%%%%%%%%%%%%%%%%%%%%%%%%%%%%%%%%%%%%%%%%%%%%%%%%%%%%%%%%%%%%%%%%%%%
%%%
%%% File: utthesis2.doc, version 2.0jab, February 2002
%%%
%%% Based on: utthesis.doc, version 2.0, January 1995
%%% =============================================
%%% Copyright (c) 1995 by Dinesh Das.  All rights reserved.
%%% This file is free and can be modified or distributed as long as
%%% you meet the following conditions:
%%%
%%% (1) This copyright notice is kept intact on all modified copies.
%%% (2) If you modify this file, you MUST NOT use the original file name.
%%%
%%% This file contains a template that can be used with the package
%%% utthesis.sty and LaTeX2e to produce a thesis that meets the requirements
%%% of the Graduate School of The University of Texas at Austin.
%%%
%%% All of the commands defined by utthesis.sty have default values (see
%%% the file utthesis.sty for these values).  Thus, theoretically, you
%%% don't need to define values for any of them; you can run this file
%%% through LaTeX2e and produce an acceptable thesis, without any text.
%%% However, you probably want to set at least some of the macros (like
%%% \thesisauthor).  In that case, replace "..." with appropriate values,
%%% and uncomment the line (by removing the leading %'s).
%%%
%%%%%%%%%%%%%%%%%%%%%%%%%%%%%%%%%%%%%%%%%%%%%%%%%%%%%%%%%%%%%%%%%%%%%%%%%%%%%

%%%%%%%%%%%%%%%%%%%%%%%%%%%%%%%%%%%%%%%%%%%%%%%%%%%%%%%%%%%%%%%%%%%%%%%%%%%%%
%%%
%%
%% This file, and the corresponding tcdthesis.sty the accompanied it, have
%% been modified for the M.Sc. styles used in Trinity College, Dublin
%%
%%
%%%%%%%%%%%%%%%%%%%%%%%%%%%%%%%%%%%%%%%%%%%%%%%%%%%%%%%%%%%%%%%%%%%%%%%%%%%%%
\documentclass[a4paper, 12pt, oneside]{report}         %% LaTeX2e document.
\usepackage {tcdthesis}              %% Preamble.
\usepackage{fancyhdr}

\mastersthesis                       %% Uncomment one of these; if you don't
% \phdthesis                         %% use either, the default is \phdthesis.

%\thesisdraft                         %% Uncomment this if you want a draft
                                     %% version; this will print a timestamp
                                     %% on each page of your thesis.

\leftchapter                         %% Uncomment one of these if you want
% \centerchapter                     %% left-justified, centered or
% \rightchapter                      %% right-justified chapter headings.
                                     %% Chapter headings includes the
                                     %% Contents, Acknowledgments, Lists
                                     %% of Tables and Figures and the Vita.
                                     %% The default is \centerchapter.

% \singlespace                       %% Uncomment one of these if you want
\oneandhalfspace                     %% single-spacing, space-and-a-half
% \doublespace                       %% or double-spacing; the default is
                                     %% \oneandhalfspace, which is the
                                     %% minimum spacing accepted by the
                                     %% Graduate School.

\renewcommand{\thesisauthor}{Qiguang Wang}                %% Your official TCD name.
\renewcommand{\thesismonth}{August}                   %% Your month of graduation.
\renewcommand{\thesisyear}{2023}                      %% Your year of graduation.
\renewcommand{\thesistitle}{Effects of PQC on Wireguard protocol}          %% The title of your thesis; use mixed-case.

\renewcommand{\thesissupervisor}{Stephen Farrell}            %% Your thesis supervisor; use mixed-case and don't use any titles or degrees.
% \renewcommand{\thesiscosupervisor}{}                %% Your PhD. thesis co-supervisor; if any.

% \renewcommand{\thesiscommitteemembera}{}
% \renewcommand{\thesiscommitteememberb}{}
% \renewcommand{\thesiscommitteememberc}{}
% \renewcommand{\thesiscommitteememberd}{}
% \renewcommand{\thesiscommitteemembere}{}
% \renewcommand{\thesiscommitteememberf}{}
% \renewcommand{\thesiscommitteememberg}{}
% \renewcommand{\thesiscommitteememberh}{}
% \renewcommand{\thesiscommitteememberi}{}


\renewcommand{\thesisauthoraddress}{...}

\renewcommand{\thesisdedication}{...}     %% Your dedication, if you have one; use "\\" for linebreaks.


%%%%%%%%%%%%%%%%%%%%%%%%%%%%%%%%%%%%%%%%%%%%%%%%%%%%%%%%%%%%%%%%%%%%%%%%%%%%%
%%%
%%% The following commands are all optional, but useful if your requirements
%%% are different from the default values in tcdthesis.sty.  To use them,
%%% simply uncomment (remove the leading %) the line(s).

\renewcommand{\thesisdegree}{Master of Science in Computer Science}
                                     %% default is "DOCTOR OF PHILOSOPHY"
                                     %% for \phdthesis or "MASTER OF ARTS"
                                     %% for \mastersthesis.  Provide the
                                     %% correct FULL OFFICIAL name of
                                     %% the degree.
\renewcommand{\thesisdegreestream}{ (Future Network System)}
                                     %% Default is empty. This is used on
                                     %% the title page of the thesis.

\renewcommand{\thesisdegreeabbreviation}{M.Sc.}
                                     %% Use this if you also use the above
                                     %% command; provide the OFFICIAL
                                     %% abbreviation of your thesis degree.
\renewcommand{\thesistype}{Dissertation}    %% Use this ONLY if your thesis type
                                     %% is NOT "Thesis" for \phdthesis
                                     %% or \mastersthesis.
                                     %% Provide the OFFICIAL type of the
                                     %% thesis; use mixed-case.

%%%
%%%%%%%%%%%%%%%%%%%%%%%%%%%%%%%%%%%%%%%%%%%%%%%%%%%%%%%%%%%%%%%%%%%%%%%%%%%%%

\usepackage{graphicx,color}
\usepackage{anysize}
\usepackage{amsmath}
\usepackage{natbib}
\usepackage{caption}
\usepackage{hyperref}
\usepackage{listings}
\usepackage{float}
%%%%%add pkg
\usepackage[mode=buildnew]{standalone}
\usepackage{tikz}
\usepackage[smartEllipses]{markdown}
\usepackage{tcolorbox}
\usepackage{multirow}
%%------------------------------------------------
%% Listing macros
%%------------------------------------------------
%% Examples for the commands in the document below
%%
%% includecode:
%% \includecode{caption for table of listings}{caption for reader}{filename}
%% - includes a file with code and adds a caption that should describe the code in some detail and a shorter caption for the table of listings
\newcommand{\includecode}[4]{\lstinputlisting[floatplacement=H, caption={[#1]#2}, captionpos=b, frame=single, label={#3}]{#4}}

%%------------------------------------------------
%% Image macros
%%------------------------------------------------

%% includescalefigure:
%% \includescalefigure{label}{short caption}{long caption}{scale}{filename}
%% - includes a figure with a given label, a short caption for the table of contents and a longer caption that describes the figure in some detail and a scale factor 'scale'
\newcommand{\includescalefigure}[5]{
\begin{figure}[htb]
\centering
\includegraphics[width=#4\linewidth]{#5}
\captionsetup{width=.8\linewidth} 
\caption[#2]{#3}
\label{#1}
\end{figure}
}

%% includefigure:
%% \includefigure{label}{short caption}{long caption}{filename}
%% - includes a figure with a given label, a short caption for the table of contents and a longer caption that describes the figure in some detail
\newcommand{\includefigure}[4]{
\begin{figure}[htb]
\centering
\includegraphics{#4}
\captionsetup{width=.8\linewidth} 
\caption[#2]{#3}
\label{#1}
\end{figure}
}





\begin{document}                                  %% BEGIN THE DOCUMENT

\thesistitlepage                                  %% Generate the title page.

%\hypersetup{pageanchor=false}
%\thesisdeclarationpage                            %% Generate the declaration page.

%\thesispermissionpage                             %% Generate the copyright permission page
%\hypersetup{pageanchor=true}

\begin{thesisabstract}                            %% the abstract for your thesis
This paper introduces post-quantum WireGuard, a post-quantum instantiation of the WireGuard protocol. The study is based on the KEMTLS protocol (ACM CCS 2020), which uses key-encapsulation mechanisms (KEMs) instead of post quantum signatures to authenticate communication parties. It brings both the TLS and the traditional post-quantum TLS into a more efficient protocol. Notably, this variant does not merely put focus on few aspects of security such as authentication and forward secrecy, as commonly pursued by earlier research at designing post-quantum protocols. Instead, it also offer bilateral post-quantum authentication with moderate cost of computation and communication. To accomplish this, we substitute the existing Elliptic Curve Diffie-Hellman key exchange with a new asymmetric crypto primitive, namely KEM. This thesis explores various combinations of different KEMs to alter existing WireGuard implementations. We propose three incremental modifications, each providing different level of security against quantum threats. 
\end{thesisabstract}

%\thesisdedicationpage                            %% Generate the dedication page.

\begin{thesisacknowledgments}                     %% Use this to write your
  Thank you Mum \& Dad.                           %% acknowledgments; it can be anything
\end{thesisacknowledgments}                       %% allowed in LaTeX2e par-mode.
  
  
\tableofcontents                                  %% Generate table of contents.
\listoftables                                     %% Uncomment this to generate list of tables.
\listoffigures                                    %% Uncomment this to generate list of figures.

%%
%% Include thesis chapters here...
%%
\chapter{Introduction}

WireGuard is an innovative secure network protocol that functions as a virtual network interface within the Linux kernel in most cases. Its primary goal is to serve as a drop-in replacement for similar secure network protocol including IPsec and OpenVPN, providing enhanced security, better performance and improved friendliness. The secure tunnel is established by binding the peer's public key with its corresponding source address. The design of WireGuard is directly deriving from the Noise framework, bearing in mind the problem of pervasive surveillance and active tampering. Also, to get over the bandwidth and computational cost , WireGuard streamline the handshake process, utilizing a single round trip to achieve security of both perfect forward secrecy (PFS) and key-compromise impersonation (K-CI) resistance. But, all these security properties only holds in the classical computer model. With the arrival of the quantum computer, it is necessary to reimage of WireGuard Diffie-Hellman (DH) style handshake to use key encapsulation mechanisms (KEMs) for implicit authentication, rather than pre-shared static key for explicit authentication. 
This research proposes a more efficient and safer way to integrate the kem algorithm into the WireGuard protocol, in a similar manner to the kem-tls and kemtlspdk and achieve both post-quantum authentication and stronger forward secrecy. The research also instantiate with a variety of post-quantum secure KEMs based on fundamentally different mathematical assumptions. 
\section{Motivation}
\label{sec:StyleOfEnglish}
\subsection{WireGuard}
WireGuard is an upcoming project to replace IPSec with a newer more modern and secure VPN protocol. It mainly lives inside the kernel , but cross platform implementations are also available. It provides a very simple and novel interface for setting up secure encrypted network tunnels. All the cryptography primitives are cutting edge: Curve25519, ChaCha20, and Poly1305 and the complexity, as well as the sheer amount of code is ignorable. Unlike TLS, which offers backward compatibility and cryptography agility. The WireGuard hard-codes all the cryptography primitive choices. The decision to hard-code its cryptography primitives was driven by a desire for simplicity and consistency. It's one of the reasons why WireGuard has gained a reputation for performance and reliability. In the meantime, these primitives provide unparalleled degree of performance in the first place, introduce fewer dependencies and less complexity to manage which make it easier to ensure that the new design is secure and that all components of the protocol work correctly together. Furthermore, Its simplicity and minimalist design philosophy aligns well with the current state of post-quantum cryptography. Many post-quantum cryptography algorithms are still being tested and optimized, and are often more computationally intensive or require larger key sizes than their classical counterparts. Having a lightweight, efficient protocol like WireGuard as the starting point could be advantageous in balancing these new algorithms' overhead and maintaining a high-performance, secure VPN. 
\subsection{Post-Quantum Cryptography}
In recent years, there has been a considerable amount of research on quantum computers which can solve specific mathematical problems that are difficult for classical computers by exploiting quantum phenomena. In the foreseeable future, the large-scale quantum computer probably will bring unprecedented speed and power in computing to reality. However, this quantum edge also poses serious problems which now is so-called **quantum apocalypse** - "store now, decrypt later" within the field of post-quantum cryptography. One of such threats lies in the domain of public key cryptography. When it comes time, the quantum computer could potentially break many of the cryptography systems, which were designed and secured under the premises of classical computing. It's crucial that we adopt a proactive approach and remain one step ahead of the potential risks. We can not afford to idle until the quantum computers begin dismantling today’s public-key cryptography is running. Therefore, the goal of this project is to scrutinize the vulnerable part of the WireGuard and propose a new WireGuard which is post-quantum safe while still keeps the virtue of original WireGuard protocol. 
The necessity for quantum safe cryptography stems from the capability of Shor's algorithm's ability to efficiently resolve problems related to factorization and discrete logarithm which are the security backbones of **RSA**, **ECC** and **ECDSA** that rely on the **IFP** (integer factorization problem), the **DLP** (discrete logarithms problem) and the **ECDLP** (elliptic-curve discrete logarithm problem) respectively. By using Shor's algorithm, a k-bit (in binary) number can be factored in time of order $O(k^3)$ using a quantum computer of 5k+1 qubits. As the NIST standardization process is ongoing [1][2], this standardization doesn't assert the supremacy of one suggestion over another. Given that the NIST standardization process has already chosen the **Selected Algorithms 2022** [3], the most secure transition method will be integrate the standardized Post-Quantum Cryptography algorithms instead of waiting for national bodies to finalize PQC algorithms in which case where the risk associated with quantum broken cryptography is not acceptable.

%The following points are couple of {\it Do's \& Dont's} that I have noted down as feedback to reports over the years. The focus of this list is to encourage writers to be specific in writing reports - some of this is motivated by Strunk and White's The Elements of Style~(\cite{strunk}). Regarding reports that are submitted as part of a degree, examiners have to read and mark these reports - make it easy for these examiners to give good marks by following a number of simple points:

% \begin{description}
% 	\item [Acronyms:] Acronyms should be introduced by the words they represent followed by the acronym in capitals enclosed in brackets e.g. "...TCP (Transmission Control Protocol)..." $\Rightarrow$  "... Transmission Control Protocol (TCP)..."
% 	\item [Contractions:] I would generally suggest to avoid contractions such as "I'd", "They've", etc in reports. In some cases, they are ambiguous e.g. "I'd" $\Rightarrow$ "I would" or "I had" and can lead to misunderstandings.
% 	\item [Avoid "do":] Be specific and use specific verbs to describe actions.
% 	\item [Adverbs:] Adverbs and adjectives such as "easily", "generally", etc should be removed because they are unspecific e.g. the statement "can be easily implemented" depends very much on the developer. 
% 	\item [Articles:] "A" and "an" are indefinite articles; they should be used if the subject is unknown. "The" is a definite article; which should be used if a specific subject is referred to. For example, the subject referred to in "allocated by the coordinator" is not determined at the time of writing and so the sentences should be changed to "allocated by a coordinator".
% 	\item [Avoid brackets:] Brackets should not be used to hide sub-sentences, examples or alternatives. The problem with this use of brackets is that it is not specific and keeps the reader guessing the exact meaning that is intended. For example "... system entities (users, networks and services) through ..." should be replaced by "... system entities such as users, networks, and services through ...".
% 	\item [Figures:] Figures and graphs should have sufficient resolution; figures with low resolution appear blurred and require the reader to make assumptions.
% 	\item  [Captions:] Use captions to describe a figure or table to the reader. The reader should not be forced to search through text to find a description of a figure or table. If you do not provide an interpretation of a figure or table, the reader will make up their own interpretation and given Murphy's law, will arrive at the polar opposite of what was intended by the figure or table.
% 	\item [Backgrounds:] Backgrounds of figures and snapshots of screens should be light. Developers often use terminals or development environments with dark backgrounds. Snapshots of these terminals or developments are difficult to read when placed into a report. 
% 	\item [Titles:] Titles of section should never be followed immediately by another title e.g. a title of a chapter should be followed by text describing the content and relevance of the sections of the chapter and could then be followed by the title of the first section of the chapter.
% 	\item [Punctuation:] A statement is concluded with a period; a question with a question mark.  
% 	\item [Spellcheckers:] Use a spellchecker!
% \end{description}


\section{Goals} 
The major goal of this thesis is to add post-quantum cryptography into WireGuard, aiming to uphold perfect forward secrecy (PFS), secrecy, authenticity, and even security against active attacks. In the current version of WireGuard, there is an efficient approach [4] applying to the WireGuard protocol that can achieve transitional post-quantum security to the WireGuard via utilizing optional pre-shared key (PSK) value. This key, calculated independently of the key agreement protocol, could be used to initialize a PQ key exchange [4]  In order to address this issue, it is necessary to integrate PQ key agreements directly into the WireGuard protocol. 
While addressing these primary goals, the secondary objectives revolve around maintaining optimal performance and usability. These aspects should be preserved as effectively as possible, even while achieving the main goals. Thus, this thesis strives to keep a balance between advancing WireGuard's security measures to confront quantum threats, without compromising its hallmark simplicity and performance.
% The arranging of figures in Latex can lead to spending a lot of time on minor issues e.g. positioning a figure in a specific location on a page, fixing minor issues with an exact size of a figure, etc. Figure~\ref{fig:ImageOfAChick} provides a simple example that demonstrates the use of one of two macros for handling figures, called {\it includefigure}; the other macro,  {\it includescalefigure}, is demonstrated in chapter~\ref{chap:Evaluation}. Figures should always be readable without magnification when printed and the resolution of an image should be sufficient to provide a clear picture when printed.

% \includefigure{fig:ImageOfAChick}{An Image of a chick}{A caption should describe the figure to the reader and explain to the reader the meaning of the figure. A Sub-clause of Murphy's Law: If the interpretation of a figure is left to a reader, the reader will misinterpret the figure, feel insulted or decide to ignore it. Do not leave it up to the reader!}{image.png}


\section{Contribution}

In this thesis, we introduce KEM-WireGuard, a generic post-quantum augmentation to the WireGuard protocol. This innovation not only improves the security design to keep secret against quantum attacks, as done in previous attempts at transitioning to post-quantum security, but also ambitiously aims for comprehensive post-quantum security, encompassing authentication as well. We also present the KEM-based instantiation of post-quantum WireGuard using Golang native cryptographic library - CIRCL (Cloudflare Interoperable Reusable Cryptographic Library). We focus on a coherent and cohesive implementation while considering the unique characteristics of different KEMs. We employ various combinations of KEMs to maximize the utilization of their unique performance and key size characteristics, with the ultimate goal of achieving optimal bandwidth and performance.
Our approach involves benchmarking and comparing various instantiations of the post-quantum WireGuard variants with previous implementations of PQ-WireGuard and the original WireGuard. A important consideration is to align as closely as possible to the original WireGuard protocol in terms of its security, performance metrics and handshake flow. Consequently, it should :
\begin{itemize}
\item Attain all the safety characteristics of WireGuard while also being resilient against attacks by a large-scale quantum computer
\item Reach a firm decision on the cryptographic primitives explicitly, thereby eliminating the need for an cryptographic negotiation.
\item Reduce the on-wire format size as well as the memory consumption.
\end{itemize}
The original WireGuard protocol heavily relies on the Elliptic Curve Diffie-Hellman key exchange, which is not a simple task to substitute with post-quantum ones. It's important to note that while the application of post-quantum algorithms in the WireGuard protocol may increase security margin , it's crucial to carefully evaluate their implementation and performance to ensure they meet the necessary standards for secure communication.  Given that SIDH/SIKE can be compromised in classical polynomial time, it's definitely left out of the consideration. The only feasible post-quantum  key exchange is CSIDH, but its security is relatively new and unproven. So we decided to follow the results of previous research on kem, modifying the WireGuard protocol to use interactive key-encapsulation mechanisms (KEMs) exclusively.
[todo]
The security of WireGuard is upheld by Donenfeld and Milner's symbolic proof and the computational proof provided by Dowling and Paterson. Although the symbolic proof encompasses a wider range of security properties and is computer verified, the computational proof, when correct, ensures robust security assurances as it relies on fewer idealized assumptions. We adapt both proofs to the PQ-WireGuard scenario, thereby maintaining WireGuard's level of security guarantees. During this process, we identify and rectify a few minor errors in the computational proof. To facilitate a standalone proof of the handshake, we incorporate an explicit key confirmation message into the PQ-WireGuard handshake as recommended.

\section{Structure}
\begin{description}

\item Chapter 1: Introduction

1.1 Background and Motivation \\
1.2 Problem Statement\\
1.3 Objectives\\
1.4 Overview of WireGuard\\
1.5 Structure of the Thesis

\item Chapter 2: Literature Review and Preliminary Concepts

2.1 Review of Post-Quantum Cryptography\\
2.2 Existing Post-Quantum VPN Software\\
2.3 Key Algorithms, Definitions, Protocols, and Cryptographic Primitives\\
2.4 Challenges in Implementing Post-Quantum Cryptography in VPNs

\item Chapter 3: Proposed Changes to WireGuard

3.1 Design Goals of PQ-WireGuard\\
3.2 From Diffie-Hellman to Key-Encapsulation Mechanisms (KEMs)\\
3.3 PQ-WireGuard Handshake Protocol\\
3.4 Security Analysis

\item Chapter 4: Implementation Decisions

4.1 Choice of Cryptographic Primitives 
4.2 Dealing with Challenges and Limitations 
4.3 Prototype Implementation and Testing

\item Chapter 5: Critical Analysis and Evaluation

5.1 Performance Analysis: PQ-WireGuard vs. WireGuard 
5.2 Analysis of Security Features 
5.3 Future Quantum Threat Model Analysis 
5.4 Average Time to Perform the Handshake

\item Chapter 6: Conclusion and Future Work

6.1 Summary of Findings\\
6.2 Implications and Significance\\
6.3 Limitations\\
6.4 Recommendations for Future Research

\item References:** [A list of references cited throughout the thesis.]

\item Appendices:** [Any supplementary material.]
\end{description}
\chapter{State of the Art}

At the beginning of each chapter, a description should introduce the reader to the content of the chapter. The description should explain to the reader the layout of the chapter, the contribution that the chapter makes to the overall dissertation and the contribution of the individual sections towards the overall chapter.

From the perspective of this document forming part of your degree, this chapter should demonstrate to the reader your knowledge of the area of your dissertation project. It should present your knowledge in a coherent and detailed form. The reader should understand that you have in-depth knowledge of the area of the dissertation without being overloaded with information.

\section{Background}

A section on the background of the dissertation should provide the reader with an introduction to existing technologies and concepts that form the basis of the
work presented in the dissertation.


\section{Closely-Related Work}

Work in research areas tends to address a number of specific aspects. Ideally, the discussion of published research should focus on the aspects that have been addressed by various publications - and not a discussion of the individual publications.

For example, if the topic would be a discussion of work on programming languages, the subsections of the related work could be discussions of object orientation and its realisation in various languages or the use of lambda functions by these languages.

\subsection{Aspect \#1}


\subsection{Aspect \#2}

\section{Summary}

Summarize the chapter and present a comparison of the projects that you reviewed.

\begin{table}[!h]
\begin{center}
	\begin{tabular}{|l|c|c|} 
	\hline
 	\bf  & \bf Aspect \#1  & \bf Aspect \#2 \\
  	\hline
	Row 1 & Item 1 & Item 2 \\
	Row 2 & Item 1 & Item 2 \\
	Row 3 & Item 1 & Item 2 \\
	Row 4 & Item 1 & Item 2 \\
	\hline
	\end{tabular}
\end{center}
\caption[Comparison of Closely-Related Projects]{Caption that explains the table to the reader}	
\label{tab:SummaryProjects}
\end{table}

\chapter{Design}
\label{chap:Design}

As outlined in Sections I and II, the WireGuard handshake is heavily based on DH, which does not have an efficient and well established post-quantum equivalent. Hence, in this section we describe how we replace DH by KEMs, for which well-established, efficient post-quantum instantiations exist.
We start by considering a simplified view on the core of the DH-based WireGuard handshake.


\section{Secuirt Aspect}
\label{sec:ProblemFormulation}
The primary goal of the adaptations proposed in this thesis is to provide security against post quantum attackers, who already are probably recording traffic at the current time for later decryption. Therefore, the necessary security properties that need to be guaranteed are listed below

\subsection{Secrecy}
Three key encapsulations using the keypairs \texttt{sski}/\texttt{spki},
\texttt{sskr}/\texttt{spkr}, and \texttt{eski}/\texttt{epki} provide
secrecy. Their respective ciphertexts are called \texttt{scti},
\texttt{sctr}, and \texttt{ectr} and the resulting keys are called
\texttt{spti}, \texttt{sptr}, \texttt{epti}. A single secure
encapsulation is sufficient to provide secrecy. We use two different
KEMs: Kyber and Classic McEliece.

\subsection{Authenticity}
The key encapsulation using the keypair \texttt{sskr}/\texttt{spkr}
authenticates the responder from the perspective of the initiator. The KEM encapsulation \texttt{sski}/\texttt{spki} authenticates the
initiator from the perspective of the responder. Authenticity is based
on the security of Classic McEliece alone.
\subsection{Pre-Shared Symmetric Key}
We allow the use of a pre-shared key (\texttt{psk}) as protocol input.
Even if all asymmetric security primitives turn out to be insecure,
providing a secure \texttt{psk} will have KEM-WireGuard authenticate both
peers, and output a secure shared key.
\subsection{Forward Secrecy}
Forward secrecy refers to secrecy of past sessions in case all static
keys are leaked. Imagine an attacker recording the network messages sent
between two devices, developing an interest in some particular exchange,
~and stealing both computers in an attempt to decrypt that conversation.
By stealing the hardware, the attacker gains access to \texttt{sski},
\texttt{sskr}, and the symmetric secret \texttt{psk}. Since the
ephemeral keypair \texttt{eski}/\texttt{epki} is generated on the fly
and deleted after the execution of the protocol, it cannot be recovered
by stealing the devices, and thus, KEM-WireGuard provides forward secrecy.
Forward secrecy relies on the security of Kyber and on proper
zeroization, i.e., the implementation must erase all temporary
variables.
\subsection{Security against State Disruption Attacks}

Both WireGuard and PQ-WireGuard are vulnerable to state disruption attacks; they rely
on a timestamp to protect against replay of the first protocol message.
An attacker who can tamper with the local time of the protocol initiator
can inhibit future handshakes, rendering the initiator's static keypair practically useless. Due to the use of the insecure NTP
protocol, real-world deployments are vulnerable to this attack. Lacking a reliable way to detect retransmission,
we remove the replay protection mechanism and store the responder state
in an encrypted cookie called ``the biscuit'' instead. Since the
responder does not store any session-dependent state until the initiator
is interactively authenticated, there is no state to disrupt in an
attack.

Note that while KEM-WireGuard is secure against state disruption, using it
does not protect WireGuard against the attack. Therefore, the hybrid
KEM-WireGuard/WireGuard setup recommended for deployment is still
vulnerable.

\subsection{Cryptographic Building Blocks}
All symmetric keys and hash values used in KEM-WireGuard are 32 bytes long.
\subsubsection{Hash}

A keyed hash function with one 32-byte input, one variable-size input,
and one 32-byte output. As keyed hash function we use the HMAC
construction {[}@rfc\_hmac{]} with BLAKE2s {[}@rfc\_blake2{]} as the
inner hash function.
\definecolor{gray}{rgb}{0.5,0.5,0.5}
\begin{tcolorbox}[colback=gray!]
\NormalTok{hash(key, data) {-}\textgreater{} key}
\end{tcolorbox}

\subsubsection{XAEAD}

Authenticated encryption with additional data for use with random
nonces. We use XChaCha20Poly1305 {[}@draft\_xchachapoly{]} in the
implementation, a construction also used by WireGuard.


\begin{tcolorbox}[colback=gray!]
\NormalTok{hash(key, data) {-}\textgreater{} key}
\NormalTok{XAEAD::enc(key, nonce, plaintext, additional\_data) {-}\textgreater{} ciphertext}
\NormalTok{XAEAD::dec(key, nonce, ciphertext, additional\_data) {-}\textgreater{} plaintext}
\end{tcolorbox}
\subsubsection{SKEM}
'Key Encapsulation Mechanism'(KEM) is the name of an interface widely
used in post-quantum-secure protocols. KEMs can be seen as asymmetric
encryption specifically for symmetric keys. KEM-WireGuard uses two different
KEMs. SKEM is the key encapsulation mechanism used with the static
keypairs in KEM-WireGuard. The public keys of these keypairs are not
transmitted over the wire during the protocol. We use Classic McEliece
460896 {[}@mceliece{]} which claims to be as hard to break as 192-bit
AES. As one of the oldest post-quantum-secure KEMs, it enjoys wide trust
among cryptographers, but it has not been chosen for standardization by
NIST. Its ciphertexts and private keys are small (188 bytes and 13568
bytes), and its public keys are large (524160 bytes). This fits our use
case: public keys are exchanged out-of-band, and only the small
ciphertexts have to be transmitted during the handshake.


\begin{tcolorbox}[colback=gray!]
\NormalTok{SKEM::enc(public\_key) {-}\textgreater{} (ciphertext, shared\_key)}
\NormalTok{SKEM::dec(secret\_key, ciphertext) {-}\textgreater{} shared\_key}
\end{tcolorbox}

\subsubsection{EKEM}

Key encapsulation mechanism used with the ephemeral KEM keypairs in
KEM-WireGuard. The public keys of these keypairs need to be transmitted over
the wire during the protocol. We use Kyber-512 {[}@kyber{]}, which has
been selected in the NIST post-quantum cryptography competition and
claims to be as hard to break as 128-bit AES. Its ciphertexts, public
keys, and private keys are 768, 800, and 1632 bytes long, respectively,
providing a good balance for our use case as both a public key and a
ciphertext have to be transmitted during the handshake.


\begin{tcolorbox}[colback=gray!]
\NormalTok{EKEM::enc(public\_key) {-}\textgreater{} (ciphertext, shared\_key)}
\NormalTok{EKEM::dec(secret\_key, ciphertext) {-}\textgreater{} shared\_key}
\end{tcolorbox}

Using a combination of two KEMs -- Classic McEliece for static keys and
Kyber for ephemeral keys -- results in large static public keys, but
allows us to fit all network messages into a single IPv6 frame.

KEM-WireGuard uses libsodium {[}@libsodium{]} as cryptographic backend for
hash, AEAD, and XAEAD, and liboqs {[}@liboqs{]} for the
post-quantum-secure KEMs.


\section{Overview of the Design}
\label{sec:OverviewOfDesign}
A description of the approach that addresses the problem identified above.
\begin{figure}[!htb]
\centering
\captionsetup{justification=centering}
\includestandalone{kem_protocol}
\caption{Generic KEM-WireGuard deisgn}
\end{figure}

\section{Summary}
\label{sec:SummaryDesign}

Every chapter aside from the first and last chapter should conclude with a summary. 
\chapter{Implementation}
\lstset{language=Python, captionpos=b, frame=single}
\captionsetup{width=.8\linewidth} 


Guess what? At the beginning of each chapter, a description should introduce the reader to the content of the chapter. The description should explain to the reader the layout of the chapter, the contribution that the chapter makes to the overall dissertation and the contribution of the individual sections towards the overall chapter.


\section{Overview of the Solution}

%% Short caption for the table of listings - long caption for the explanation for the reader
\includecode{Sample Code}{Lengthy caption explaining the code to the reader}{lst:snippet}{snippet.py}

The code in listing~\ref{lst:snippet} is a demonstration how to include a file with code into the template.



\section{Component One}

%% Defaults for listings

The code in listing~\ref{lst:snippet2} is a demonstration how to include code in the template.

%% Short caption for the table of listings - long caption for the explanation for the reader
\begin{lstlisting}[caption={[Sample Code 2]Second Lengthy caption}, label={lst:snippet2}]
x = 1
if x == 1:
    # indented four spaces
    print("x is 1.")
\end{lstlisting}


\section{Summary}

Every chapter aside from the first and last chapter should conclude with a summary. 


\chapter{Evaluation}
\label{chap:Evaluation}

As the main question here is one of usability, the following will mainly be a performance comparison, comparing the proof-of-concept handshake protocol implementations to the default WireGuard protocol, as implemented in WireGuard-Go. To accommodate for any of the three higher security levels defined in the Feature Specification, only the handshake code was changed. Therefore, most focus will be put on the handshake computations, as well as initiation and response message transmissions.
\subsection{Message Sizes}
Adding more key exchange primitives to the handshake protocol obviously adds addi- tional data, in the form of public keys and encapsulated values. This data needs to be transmitted in the initiation or response message.
With PQ KEMs that data is relatively large for network transmissions, hundreds of Bytes up to hundreds of Kilobytes in the worst cases. Thus, packets may easily become larger than the Ethernet maximum transmission unit (MTU) of 1500 Bytes.
Since WireGuard is based on UDP, there is no segmentation functionality on the Transport Layer. If large data is sent over UDP, packet fragmentation may happen on the IP layer. Relying on this functionality is considered fragile, and the IETF specifically advises against it. [8] This is because the IP standard allows routers to drop large packets silently.
A way around IP fragmentation is to split the datagrams on the application layer.
When using UDP, this can still be terrible for performance because the handshake will fail anytime one of the datagrams is lost. The more datagrams we need for one message, the higher the probability for losing one of them is. Thus, breaking up messages into more datagrams makes the handshake take longer on average.
A datagram size of 1436 Bytes is reasonable for this, because according to analysis by Shannon and Moore [31] around 98% of MTUs are between 1484 1500 Bytes.
Subtracting 8 Bytes for the UDP header and 40 Bytes for the possibility of an IPv6 header, we arrive at the number above. Also, the IPv6 standard recommends an MTU of at least 1500 Bytes, and requires acceptance of packets up to that same size. 
Choosing cryptographic primitives with small key sizes is thus essential to prevent unnecessary overhead, in the form of excessive transmission times. Cryptographic primitives with excessively large public keys or encapsulated values have thus been excluded from all following considerations. For example, all code-based cryptosystems, which need many Kilobytes for a public key.

\begin{table}[!h]
\begin{center}
\begin{tabular}{|l|l|l|l|}
\hline
Primitive        & PublicKeySize & PrivateKeySize & CipherTextSize \\ \hline
X25519           & 32            & 32             & 32             \\ \hline
Kyber512         & 800           & 1632           & 768            \\ \hline
NTRU LPRime      & 897           & 164            & 993            \\ \hline
Streamlined NTRU & 994           & 1518           & 897            \\ \hline
BIKE-L1          & 1541          & 5223           & 1573           \\ \hline
HQC-128          & 2249          & 2289           & 4481           \\ \hline
McEliece348864   & 261120        & 6492           & 96             \\ \hline
\end{tabular}
\end{center}
\caption{Bytes needed for the NIST-PQC level I primitives}
\end{table}
\subsection{Individual runtime}
\begin{table}[!h]
\begin{center}
\begin{tabular}{|l|l|l|l|}
\hline
Runtime  & Key Generation & Encapsulation & Decapsulation \\ \hline
mceliece348864f          & 51.88          & 0.35          & 19.70         \\ \hline
mceliece460896f          & 166.49         & 0.66          & 43.96         \\ \hline
kyber512                 & 0.02           & 0.01          & 0.01          \\ \hline
kyber768                 & 0.03           & 0.02          & 0.02          \\ \hline
ntrulpr653               & 5.74           & 11.45         & 17.09         \\ \hline
sntrup653                & 35.66          & 5.72          & 17.16         \\ \hline
BIKE-L1                  & 0.13           & 0.03          & 0.5           \\ \hline
HQC-128                  & 0.03           & 0.05          & 0.09          \\ \hline
\end{tabular}
\end{center}
\caption{Runtime for the NIST-PQC level I primitives (in millisecond)}
\end{table}
\section{Experiments}
In the case where experiments have been carried out, the experimental setup and the values that were defined for the variables need to be presented in a table e.g. table~\ref{tab:experimentsetup}.




\section{Results}
\begin{table}[!h]
\begin{center}
\begin{tabular}{|l|r|r|rrr|}
\hline
\multirow{2}{*}{Protocol}                                                            & \multicolumn{1}{l|}{\multirow{2}{*}{\begin{tabular}[c]{@{}l@{}}Initiation size\\ (bytes)\end{tabular}}} & \multicolumn{1}{l|}{\multirow{2}{*}{\begin{tabular}[c]{@{}l@{}}Response size\\ (bytes)\end{tabular}}} & \multicolumn{3}{l|}{Time in microseconds}                                              \\ \cline{4-6} 
                                                                                     & \multicolumn{1}{l|}{}                                                                                   & \multicolumn{1}{l|}{}                                                                                 & \multicolumn{1}{l|}{Server} & \multicolumn{1}{l|}{Client} & \multicolumn{1}{l|}{Total} \\ \hline
\begin{tabular}[c]{@{}l@{}}WireGuard\\ (original)\end{tabular}                       & 148                                                                                                     & 92                                                                                                    & \multicolumn{1}{r|}{0.36}   & \multicolumn{1}{r|}{0.47}   & 1.37                       \\ \hline
\begin{tabular}[c]{@{}l@{}}KEM-WireGuard\\ (kyber512 only/time-optimal)\end{tabular} & 1700                                                                                                    & 1596                                                                                                  & \multicolumn{1}{r|}{0.21}   & \multicolumn{1}{r|}{0.34}   & 0.57                       \\ \hline
\begin{tabular}[c]{@{}l@{}}KEM-WireGuard\\ (mcliece348864f only)\end{tabular}        & 261348                                                                                                  & 252                                                                                                   & \multicolumn{1}{r|}{20.82}  & \multicolumn{1}{r|}{91.47}  & 133.36                     \\ \hline
\begin{tabular}[c]{@{}l@{}}KEM-WireGuard\\ (ntrulpr653 only)\end{tabular}            & 2054                                                                                                    & 2110                                                                                                  & \multicolumn{1}{r|}{39.88}  & \multicolumn{1}{r|}{51.53}  & 91.40                      \\ \hline
\begin{tabular}[c]{@{}l@{}}KEM-WireGuard\\ (sntrup653 only)\end{tabular}             & 2023                                                                                                    & 1854                                                                                                  & \multicolumn{1}{r|}{28.64}  & \multicolumn{1}{r|}{74.67}  & 103.89                     \\ \hline
\begin{tabular}[c]{@{}l@{}}KEM-WireGuard\\ (BIKE-L1)\end{tabular}                    & 3246                                                                                                    & 3206                                                                                                  & \multicolumn{1}{r|}{1.46}   & \multicolumn{1}{r|}{3.65}   & 4.60                       \\ \hline
\begin{tabular}[c]{@{}l@{}}KEM-WireGuard\\ (HQC-128)\end{tabular}                    & 6862                                                                                                    & 9022                                                                                                  & \multicolumn{1}{r|}{0.49}   & \multicolumn{1}{r|}{0.61}   & 0.84                       \\ \hline
\begin{tabular}[c]{@{}l@{}}KEM-WireGuard\\ (M\&K/size-optimal)\end{tabular}          & 1028                                                                                                    & 252                                                                                                   & \multicolumn{1}{r|}{20.40}  & \multicolumn{1}{r|}{20.95}  & 41.47                      \\ \hline
\end{tabular}
\end{center}
\caption{Runtime for the NIST-PQC level I primitives}
\end{table}
Figures that present results such as figure~\ref{fig:measurements} need to display descriptions of the axes, the units and scales of the measurements, statistical values, etc. Where measurements were taken from experiments, error bars or confidence intervals need to be provided to give the reader an indication of the spread of the measurements.

\includescalefigure{fig:measurements}{Measurement of System Wakeups}{Long caption that describes the figure to the reader}{1}{measurements.png}


\section{Summary}

Every chapter aside from the first and last chapter should conclude with a summary that presents the outcome of the chapter in a short, accessible form. 
\chapter{Conclusions \& Future Work}
\label{chap:Conclusions}

This chapter should summarize the work presented in the dissertation and discuss the conclusions that can be drawn from the work and the results presented in chapter~\ref{chap:Evaluation}.


\section{Future Work}

The section may present a list of items that were beyond the scope of the dissertation.

% \begin{thebibliography}{refs}                   %% Start your bibliography here; you can
\addcontentsline {toc}{chapter}{Bibliography}     %% Force Bibliography to appear in contents
\bibliographystyle{apalike}
\bibliography{refs}                               %% also use the \bibliography command
%\end{thebibliography}                            %% to generate your bibliography.


%\addcontentsline {toc}{chapter}{Appendices}       %% Force Appendices to appear in contents
%\begin{appendix}
%\chapter*{Appendix}

...

% \include{appendix2}
%\end{appendix}




\end{document}                                    %% END THE DOCUMENT
